% set counter to n-1:
\setcounter{chapter}{0}

\chapter{Introduction}

The price for breast enhancement surgery was estimated to be \$3718 in 2017 according to the American Society of Plastic Surgery\footnote{https://www.plasticsurgery.org/cosmetic-procedures/breast-augmentation/cost}. Next to the financial aspects there are also risks connected to undergoing surgery and also not knowing exactly what the result will look like. Before commiting to this kind of operation, it should be possible to generate a preview of the outcome from a few images. This thesis aims to design a method that is able to predict a 3D model of the outcome by learning a mapping between paramteric models. Additionally, the idea is explored if it is possible to generate a parametric model from a character modelling software.

%https://www.plasticsurgery.org/cosmetic-procedures/breast-augmentation/cost



\section{Parametric Model}
A previous implementation by Biland \cite{Biland17} was used to create parametric models. A parametric model can describe all data that went into the model with its parameters. For example, the physical appearance of a person can be roughly described by their height, skin tone and hair color, where these three are the parameters of this parametric model. This is of course only an approximation as the description of the person would increase with more parameters. It is also possible that one parameter influences multiple features. In the previous example, when the height of a person is raised, the length of the arms is also proportionally increased.

\section{Mapping}
A mapping between sets associates each element in the first set with one or more elements of the second set. An example for a simple mapping could be the numbers one to twenty-six as the first set and the letters of the alphabet as the second. In the case of two paramteric models, the goal is to find a mapping that describes the relationship between the parameters of the first and second model. This mapping can either be linear or non-linear. The difference between a linearity and a non-linearity can be described with a simple example. The time it takes to drive $10km$ in a car at $10\frac{km}{h}$ is $1h$. If the distance to drive is doubled to $20km$, so will the time it takes. That is because distance to drive and time it requires are in a linear relationship. On the other hand, given that the braking distance while travelling with $10\frac{km}{h}$ is $1m$, the braking distance while travelling with $20\frac{km}{h}$ is $4m$. This is due to the fact, that speed and braking distance are related in a quadratic, non-linear manner.

\section{Applications}
-Breast shape/look prediction
-Medical applications
