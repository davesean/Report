% set counter to n-1:
\setcounter{chapter}{1}

\chapter{Related Work}
In the field of modelling the human body, early on, a lot of work was done concerning the face. Jeng et al. \cite{jeng1998facial} proposed a geometric face model to localize the face in an image and detect facial features efficiently. The reason for the need of a functioning model for faces may have been the upcoming importance of facial recognition.\\
Parametric models have also been applied to reshaping the human body in images using semantic parameters. \cite{zhou2010parametric} Their approach included reshaping a morphable 3D model and applying the reshaping effects using an image warping approach. Other work was done by Allen et al. \cite{allen2003space} on parametrizing full body scans and using the parameters in a variety of applications. One application was to find features connected to certain parameters, such that for example the weight could be increased and the resulting mesh would increase in size in a meaningful way.\\
Work about specifically modelling the breast has been done by Galle et al. \cite{gallo2010human}. They proposed a parameter space based on MRI scans to describe the shape of a human breast using PCA. Ciechomski et al. \cite{de2012development} proposed a web approach to create 3D models from images where participants uploaded images and selected keypoints. These 3D models can later be accessed by surgeons to visualize different types and sizes of implants applied.
